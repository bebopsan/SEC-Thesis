\chapter{Introduction}

\note[NGZ]{La introduccion es lo que motiva al lector a leer el documento. Debe responder las preguntas: Por qu\'e es relevante esto?, Por qu\'e es relevante que yo (como ingeniero f\'isico) lo haya abordado?, Qu\'e componentes de ingenier\'ia y de f\'isica se han tocado en el trabajo?}
Preface:

\note[NGZ]{A\~nadir\'e una serie de preguntas que pueden dar respuesta a algunas de las cosas que se esperan de una Introducc\'on. Nada de esto es obligatorio y lo pongo como una gu\'ia para la escritura.}

Qu\'e se describe en este documento? y en qu\'e consisti\'o el proyecto? (Qu\'e se va a ver en este trabajo y qu\'e no se va a ver en este trabajo)


Definition of the problem will be treated in chapter \ref{ch:Problem}, followed by a brief state of the art recollection where some background in the topics of photonic crystals and computational methods for physics will be mentioned. Then, fundamental concepts behind light propagation in periodic materials (chapter \ref{ch:Electromagnetic waves in periodic media})  and the model used for the finite element procedure used are presented.

C\'omo se desarroll\'o el proyecto? (Asociado a algo metodol\'ogico)

Cu\'al es la motivaci\'on del proyecto? Cu\'al es la relevancia del proyecto?

Cu\'ales son las posibles aplicaciones de esto?

Por qu\'e te interesaste en la f\'isica computacional?

Qu\'e te llam\'o la atenci\'on del problema de los cristales fot\'onicos? (Puede ir acompa\~nado de alguno de los art\'iculos de revisi\'on).


